Данная работа была выполнена, как курсовая, для предмета ООП. Движок игры был построен с помощью библиотеки SFML С++ / SFML C\#.

\mbox{\hyperlink{namespace_cosmic_defender}{Cosmic\+Defender}} -\/ это идея, которая была взята из игры Asteroids 1972 года. В этой игре используются различные боссы, астероиды и корабли. В \mbox{\hyperlink{namespace_cosmic_defender}{Cosmic\+Defender}} я очень часто пользовался векторной математикой, в итоге получилось некая левитация в космосе, так же была разработана система двигателей, с помощью которых вектора переходили в обратные, чтобы преодолеть левитацию.

Необходимые Доработки\+: В игре была разработана гравитация, которая планировалась будет использоваться в уровне с чёрной дырой, но данный уровень находится в разработке. Так же в игре была разработана система здоровья, которая активно использовалась на боссах и на самом герое, её необходимо доработать, чтобы создать баланс игры. В начальном интерфейсе была сделана заготовка, которая должна отражать здоровье, урон и т.\+д. твоего корабля. Так же в была идея на разработку прокачки, чтобы последующие корабли открывались после какого-\/нибудь события. Есть небольшой баг, при появлении объектов в точке (0,0), например при спавне босса, его хитбокс почему-\/то закрывает весь экран и пули взрываются сразу при выстреле.

Баланс\+: Необходимо сбалансировать кол-\/во врагов, нужно чётко установить кол-\/во врагов которые будут спавниться на каждом уровне. Спавн босса происходит сразу после того, как будут убиты 90\% кораблей.

Вывод\+: -\/Игра требует небольших доработок, но впринципе ядро готово, что-\/то ввести/удалить не составит большого труда. -\/\mbox{\hyperlink{namespace_cosmic_defender}{Cosmic\+Defender}} написана в 2 экземплярах С\# и С++, но в С++ есть небольшие проблемы, так как основной проект был написан на C\#, а на С++ я портировал.

Обзор игры\+: \href{https://youtu.be/14DeaKXbjQs}{\texttt{ https\+://youtu.\+be/14\+Dea\+KXbj\+Qs}} 